\documentclass{article}
\usepackage{hyperref}
\usepackage{xcolor}
\title{Designing a Pokemon Database with SQL vs. NoSQL using ChatGPT}
\author{Tanya Singh}
\date{May 2024}

\begin{document}

\maketitle

\section{Introduction}
This report explores your experience designing a Pokemon database using both SQL and NoSQL with the assistance of ChatGPT. It compares the approaches in terms of ease of use, flexibility, and ChatGPT's helpfulness in each scenario.

\section{Designing with SQL Server}
ChatGPT provided a well-structured approach to designing the database using SQL Server.
\subsection{Tables}
\begin{itemize}
    \item Pokemon ( PokemonID (Primary Key), Name, PrimaryType, SecondaryType (nullable))
    \item Type ( TypeID (Primary Key), TypeName)
    \item Move ( MoveID (Primary Key), Name, Type (Foreign Key referencing TypeID), Power)
    \item PokemonMoves (PokemonID (Foreign Key referencing PokemonID), MoveID (Foreign Key referencing MoveID))
\end{itemize}
\subsection{Observations}
\begin{itemize}
    \item ChatGPT effectively explained the purpose of each table and its role in representing Pokemon data.
    \item The use of a separate PokemonMoves table to handle the many-to-many relationship between Pokemon and Moves was well explained.
    \item ChatGPT provided necessary constraint from the description itself and assigned primary and foreign keys.
    \item Although when firstly asked about steps to design a database, ChatGPT did talk about Normalization, but during the design process, ChatGPT did not mention it
\end{itemize}

\section{Designing with NoSQL (MongoDB)}
ChatGPT also offered a well-defined approach to using MongoDB for the Pokemon database:
\subsection{Collections and Document structures}
\begin{itemize}
    \item Pokemon: \{ ``\_id": ObjectId(), ``Name": ``Bulbasaur", ``Types": [``Grass"], ``Moves": [``Tackle", ``Vine Whip", ``Return"] \}
    \item Move: \{ ``\_id": ObjectId(), ``Name": ``Tackle", ``Type": ``Normal", ``Power": 35 \}
    \item Type: \{ ``\_id": ObjectId(), ``Name": ``Grass", ``EffectiveAgainst": [``Water"], ``WeakAgainst": [``Fire", ``Flying"] \}
    \item PokemonMoves \{``\_id": ObjectId(),
    ``PokemonID": "\le Reference to Pokémon Document \ge",
    ``MoveID": "\le Reference to Move Document \ge"\}

\end{itemize}
\subsection{Observations}
\begin{itemize}
    \item ChatGPT explained the concept of collections and documents, highlighting the flexible schema in NoSQL
    \item It gave both embedding moves within Pokemon documents and using references as options for many-to-many relationships.
    \item ChatGPT did not discuss the potential drawbacks of embedded documents, such as data duplication if a move is learned by multiple Pokemon.
\end{itemize}


\section{Analysis}
\begin{itemize}

\item \textbf{Easy to use:}
\begin{itemize}
    \item SQL Server offered a structured approach with clear steps and familiar table concepts.
    \item NoSQL provides more flexibility in schema design but might require more upfront planning for complex relationships.
\end{itemize}

\item \textbf{Flexibility:}
\begin{itemize}
    \item SQL Server uses a schema, which can be good for fixed data models but isn't flexible for evolving ones.
    \item NoSQL's flexible schema allows for easier adaptation to changing data structures.
\end{itemize}

\item \textbf{Instructing ChatGPT:}
\begin{itemize}
    \item It is better to give ChatGPT the complete database description when asking it to design
    \item When provided information about the database in steps or pieces, ChatGPT makes more mistakes
\end{itemize}

\end{itemize}

\section{Conclusion}
\subsection{Areas for Improvement}
\begin{itemize}
    \item \textbf{NoSQL considerations}: While ChatGPT effectively explained NoSQL concepts, it could be improved by providing more in-depth guidance on optimizing schema design and query performance in MongoDB

    \item \textbf{Normalization}: ChatGPT just briefly mentioned Normalization without showing the actual implementation, this could be improved.
\end{itemize}
\subsection{Learnings}
\begin{itemize}
    
\item \textbf{Data Structure Stability:} If the data structure is likely to change frequently, NoSQL might be more suitable.

\item \textbf{Data Integrity Requirements: } If data integrity and adherence to a fixed schema are critical, SQL might be a better choice.

\item \textbf{Query Complexity: } Query performance in NoSQL can depend on factors like document structure and indexing strategies. So, SQL offers better performance for complex queries requiring joins or aggregations.

\item \textbf{Potential Drawbacks: } There are challenges associated with each approach, such as scalability concerns in SQL or data consistency issues in NoSQL.

\end{itemize}
\nextline \nextline
This experience highlights the strengths and weaknesses of both SQL and NoSQL approaches and how ChatGPT works with them. Though both of them work well for this database, a complete database design might involve additional considerations beyond the scope of this exploration.
\end{document}

