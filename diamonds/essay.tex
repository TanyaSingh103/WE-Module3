\documentclass{article}
\title{Developing strategies for the bidding card game 'Diamonds' with GenAI}
\author{Tanya Singh}
\date{March 2024}

\begin{document}

\maketitle

\section{Introduction}
"Explain it like you are explaining it to a five year old", is actually harder than I initially thought. I mean, explaining a card game to a five year old would be pretty difficult, given their fleeting attention span. So thankfully, instead of a five year old, I had to explain the card game to ChatGPT.

\section{Problem Statement}
My job was to explain to ChatGPT what the game of "Diamonds" is, basically to teach the rules of the game to it and, hopefully manage to get it to write working code that can be understood by a person.

The task in a sense, was to to "teach" GenAI something that was not known to it before hand, to get it utilize bits of information from different sources to generate ideas that might be related to other things but are "new" in the given context. 


\section{Teaching GenAI the game}
Since I've experienced the trouble of AI repeating the same things over and over even when you correct it, I decided to give it an extremely detailed explanation which would cover almost everything I knew about the game.

Here's the first prompt I gave ChatGPT to explain the game:

I made sure to cover everything from the basic rules to the exception case of a tie and also when the game ends. I tried to explain the game in a sort of currency system where I described the diamonds as an item of value and the other cards as currency.

Surprisingly, ChatGPT understood everything in the first try and gave a pretty good summary of it's understanding about the game play.

I asked ChatGPT if it could come up with any strategies to which it stated a few things to keep in mind when playing. At this point, ChatGPT gave me very intelligent tactics like Opponent Observation, about how one should read the opponent's bidding patterns, bluffing and even misdirection. Though these ideas are pretty easy for a person to implement, I doubted if an AI could do the same. It even talked about early, mid and end game bidding, which again made me think about the implementation. 

I then asked ChatGPT if it had any questions about the game (not the best order of things, should have asked this before talking about strategies) and it asked me four questions, and I gave the needed clarification. The questions weren't particularly interesting but just general queries to confirm already given information.

I asked ChatGPT to play the game with me where it should keep track of the points and draw random cards for diamonds, while playing against me and that I would tell it my bids.

I won the first two bids and ChatGPT stated the reflections it made from them saying that I made bold bids to secure high-value diamond cards early in the game.(All I was doing was placing one rank more than the diamond card)

When I asked GPT what cards it had left, it answered correctly, which seems fine since there had been only two rounds played. For the third round, I just made a bid equal to the rank of the card and ChatGPT called it a "balanced approach" which seemed pretty reasonable but ChatGPT couldn't win a round yet so I asked it to discuss it's strategies it had come up with.

\section{Strategies and Code}
After playing the game once, ChatGPT added a few things to it's previous answer about strategies. Now, it had included counting cards to keep track of what has been used and what the opponent has used. This idea seemed a bit more implementable by AI. Though it was still thinking along the lines of trying to somehow trick the opponent, I think the focus slightly shifted to what it should be doing too.

\end{document} 


