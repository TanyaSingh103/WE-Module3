\documentclass{article}
\title{Developing strategies for the bidding card game 'Diamonds' with GenAI}
\author{Tanya Singh}
\date{March 2024}

\begin{document}

\maketitle

\section{Introduction}
"Explain it like you are explaining it to a five year old", is actually harder than I initially thought. I mean, explaining a card game to a five year old would be pretty difficult, given their fleeting attention span. So thankfully, instead of a five year old, I had to explain the card game to ChatGPT.

\section{Problem Statement}
My job was to explain to ChatGPT what the game of "Diamonds" is, basically to teach the rules of the game to it and, hopefully manage to get it to write working code that can be understood by a person.

The task in a sense, was to to "teach" GenAI something that was not known to it before hand, to get it utilize bits of information from different sources to generate ideas that might be related to other things but are "new" in the given context. 


\section{Teaching GenAI the game}
Since I've experienced the trouble of AI repeating the same things over and over even when you correct it, I decided to give it an extremely detailed explanation which would cover almost everything I knew about the game.

Here's the first prompt I gave ChatGPT to explain the game:

"okay so the game is called "Diamonds" It can be played by two or three players at a time. 
It is a cards game where the diamond cards are considered as an item of value and all the non-diamond cards are considered as currency. 
The value of each card corresponds to it's rank(2<3<4<5<6<7<8<9<T<J<Q<K<A). 
The players get a set of spades/clubs/hearts each (13 cards for each player) and the diamond cards are shuffled and placed face down. 
A card from the diamond pile is taken out. Each player takes a currency card from the set they have and places it as a bid. 
The players don't reveal their bids to each other until everyone has made one. When the bids are revealed, the one with higher bid gets the diamonds and bid cards are discarded. 
The person with highest bid gets points equivalent to the rank of the diamonds card. If all players have same bid, the points are divided into equal parts among them. 
If it is a three player game and two players get the same higher bid, they divide it equally and the third player doesn't get anything. 
The game goes on till all diamond cards are exhausted and the player with highest number of points wins.
"

I made sure to cover everything from the basic rules to the exception case of a tie and also when the game ends. I tried to explain the game in a sort of currency system where I described the diamonds as an item of value and the other cards as currency.
\end{document} 


