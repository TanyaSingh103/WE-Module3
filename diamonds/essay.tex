\documentclass{article}
\usepackage{hyperref}
\usepackage{xcolor}
\title{Developing strategies for the bidding card game `Diamonds' with GenAI}
\author{Tanya Singh}
\date{March 2024}

\begin{document}

\maketitle

\section{Introduction}
``Explain it like you are explaining it to a five year old", is actually harder than I initially thought. I mean, explaining a card game to a five year old would be pretty difficult, given their fleeting attention span. So thankfully, instead of a five year old, I had to explain the card game to ChatGPT.

\section{Problem Statement}
My job was to explain to ChatGPT what the game of ``Diamonds" is, basically to teach the rules of the game to it and, manage to get it to write working code that can be understood by a person. \newline

The task actually, was to to teach GenAI something that was not known to it before hand, to get it utilize bits of information from different sources and generate ideas that would be new in the given context. 

\section{Teaching GenAI the game}
Initially, I had to restart the chat when I was explaining the game to ChatGPT, my explanation was too vague and sometimes ChatGPT just won't let go of a rule it had misunderstood. It would keep repeating the same thing over and over. \newline

So I decided to give it an extremely detailed explanation which would cover nearly everything I knew about the game.\newline

I made sure to cover everything from the basic rules, to the exception case of a tie and also when the game ends. I tried to explain it as a sort of currency system, where I described the diamonds as something of value and the other cards as currency. \newline

Surprisingly, ChatGPT understood everything in the first try this time and gave a pretty good summary of it's understanding about the game play.\newline

I asked ChatGPT if it could come up with any strategies, to which it stated a few things to keep in mind when playing. At this point, ChatGPT gave me some smart tactics like opponent observation, about how one should read the opponent's bidding patterns, bluffing and even misdirection. Though these ideas are pretty easy for a person to implement, I doubted if an AI could do the same. It even talked about early, mid and end game bidding, which again made me think about the implementation. \newline

I then asked ChatGPT if it had any questions about the game (not the best order of things, should have asked this before talking about strategies) and it asked me four questions, and I provided the needed clarification. The questions weren't particularly interesting, just general queries to confirm already given information. \newline

I asked ChatGPT to play the game with me where it kept track of the points and drew random cards for diamonds, while playing against me. \newline

I won the first two bids and ChatGPT stated the reflections it made from them, saying that I made bold bids to secure high-value diamond cards early in the game.(All I was doing was placing one rank more than the diamond card) \newline

When I asked ChatGPT what cards it had left, it answered correctly, which seems fine since there had been only two rounds played. For the third round, I just made a bid equal to the rank of the card and ChatGPT called it a ``balanced approach" which seemed pretty reasonable but ChatGPT couldn't win a round yet, so I asked it to discuss it's strategies it had come up with. \newline

\section{Strategies and Code}
After playing the game once, ChatGPT added a few things to it's previous answer about strategies. Now, it had included counting cards to keep track of what had been played and what the opponent had used. This idea seemed a bit more implementable by AI. Though it was still thinking along the lines of trying to somehow trick the opponent, I think the focus slightly shifted to what it should be doing too. \newline

Then, I asked it what approach I should follow if I were to write a python program where the computer plays against the user for a two player game of diamonds. To this, ChatGPT gave me what I think was a reasonable answer where it divided the whole thing into sections and steps as it gave me a basic outline of the code. \newline

I had to ask it to tweak the code a bit to remove a few extra operations that didn't seem necessary. It took a few tries to explain why there wasn't a need to sort the displayed player's hand, since it was already in a sorted state. \newline

Although the code was alright for a first time try, I didn't see any of the strategies that ChatGPT initially recommended in the code. It had just applied a simple adaptive bidding strategy. But there was no sign of opponent observation. \newline

I then asked ChatGPT if it could come up with a better strategy, then it went back to opponent behavior and a few ideas it had discussed before, talking about implementation considerations too. \newline

I asked if it could implement the strategy in code, to which it generated a computer bid function, which used the information of past bids as well as adjusted the bids according to game progress. The computer's bid was made according to the current state of the game rather than just the diamond card that was drawn.
In a way, ChatGPT was now starting to look at the bigger picture. \newline

\section{Analysis and Conclusion}
There were a few observations I made about the conversation:

\begin{itemize}
    \item ChatGPT grasped new information quickly when given detailed information
    \item ChatGPT struggled with understanding the step where it was sorting already sorted data
    \item While ChatGPT first suggested an advanced strategy like opponent observation, it's implementation focused on simpler adaptive bidding
    \item After being directed to implement the strategy, ChatGPT was able to write code accordingly
\end{itemize}

So, it can be said that GenAI can learn games, develop strategies, and even generate playable code. But, it tends to struggle with maintaining focus on complex strategies and dealing with abstract concepts. 

\newpage

\section{Links}
    \begin{itemize}
    \color{blue}
        \item \href{https://chat.openai.com/share/dee4d319-f2da-48be-8f3f-cdcc8f603116}{Transcript for my chat}
    
        \item \href{https://colab.research.google.com/drive/13c1AyFl__FLPmxBiya7yxov_iHZq7-l7?usp=sharing}{The final code for the game}
    \end{itemize}

\end{document}


